\documentclass{article}
\topmargin -0.5in \oddsidemargin -0.25in \textheight 9in
\textwidth 6.5in
\usepackage{graphics}
\usepackage{graphicx}
\usepackage{forest}
\usepackage{tikz}
\usepackage{amsmath}
\usepackage{amssymb}
\usepackage{multirow}
\usepackage{siunitx}


\begin{document}
{\bf ICS 271}

{\bf Fall 2016}

{\bf Student ID : 26642334}

{\bf Student Name: Yu Guo}

{\bf Instructor : Kalev Kask}

{\bf Homework Assignment 5}

{\bf Due Thursday, 11/10}




\begin{enumerate}

% 1.
\item
	\begin{enumerate}
		\item 4

		Only when both $A$ and $B$ are \textit{True}, $B \land A$ would be \textit{True}. $C$ and $D$ could be any value. So there should have $1 \times 2^2 = 4$ models.

		\item 15

		$\neg A \lor \neg B \lor \neg C \lor \neg D$ 
		$= \neg(A \land B \land C \land D)$

		Only when $A,B,C,D$ are all \textit{True}, the sentences is \textit{false}. So there should have $2^4-1=15$ models.

		\item 0

		$(A \Rightarrow B) \land A \land \neg B \land C \land D$ \\
		$ = (\neg A \lor B) \land A \land \neg B \land C \land D$ \\
		$ = \neg (A \land \neg B) \land (A \land \neg B) \land C \land D$

		$\neg (A \land \neg B)$ and $(A \land \neg B)$ could NOT be \textit{True} at same time. So there's no models for this sentence.
	\end{enumerate}


% 2.
\item
	\begin{enumerate}
		\item Define: \\
		$A:$ The car is at John's house. \\
		$B:$ The car is at Fred's house. 

		\item statement 1: $A \lor B$ \\
		statement 2: $\neg B \Rightarrow A$

		\item Statement 2 is equivalent with Statement 1, so we can not determine where the car is.
	\end{enumerate}

% 3.
\item \textit{unit resolution}: 
$$\frac{\alpha \lor \beta, \:\neg \beta}{\alpha}$$

True table: 
\begin{table}[ht]
\centering
\begin{tabular}{|c|c|c|c|c|}
\hline
$\alpha$ & $\beta$ & $\alpha \lor \beta$ & $\neg \beta$ & $(\alpha \lor \beta)\land \neg \beta$ \\ \hline
T & T & T & F & F \\ \hline
\underline{T} & F & T & T & \underline{T} \\ \hline
F & T & T & F & F \\ \hline
F & F & F & T & F \\ \hline
\end{tabular}
\end{table}

From the True table, we can conclude that, when $(\alpha \lor \beta)\land \neg \beta$ is \textit{True}, $\alpha$ is \textit{True}. So the \textit{unit resolution} is SOUND.

% 4.
\item
\begin{align*}
& \neg [((P \lor \neg Q) \rightarrow R) \rightarrow (P \land Q)] \\
= & \neg [(\neg (P \lor \neg Q) \lor R) \rightarrow (P \land Q)] \\
= & \neg [\neg (\neg (P \lor \neg Q) \lor R) \lor (P \land Q)] \\
= & (\neg (P \lor \neg Q) \lor R) \land \neg (P \land Q) \\
= & ((\neg P \land Q) \lor R) \land (\neg P \lor \neg Q) \\
= & (\neg P \lor R) \land (Q \lor R) \land (\neg P \lor \neg Q) \\
\end{align*}


% 5.
\item $N$-Queen constraints: ($q_{k,l}$, $(k,l)$ means the square on $k$-th row and $l$-th column)

1. Any of two Queens should not be in the same column.
$$ \bigwedge_{k=1}^{N}\{\bigvee_{j=1}^{N}[q_{j,k} \land (\bigwedge_{\substack{i=1\\i \neq j}}^{N} \neg q_{i,k})]\} $$

2. Any of two Queens should not be in the same row.
$$ \bigwedge_{k=1}^{N}\{\bigvee_{j=1}^{N}[q_{k,j} \land (\bigwedge_{\substack{i=1\\i \neq j}}^{N} \neg q_{k,i})]\} $$

3. Any of two Queens should not be in the same diagnal line (NW to SE).
$$ \bigwedge_{k=1}^{N}\{\bigvee_{j=1}^{N}[q_{j,k} \land (\bigwedge_{\substack{i=1\\i \neq j\\1\leqslant i-j+k \leqslant N}}^{N} \neg q_{i,i-j+k})]\} $$

4. Any of two Queens should not be in the same diagnal line (NE to SW).
$$ \bigwedge_{k=1}^{N}\{\bigvee_{j=1}^{N}[q_{j,k} \land (\bigwedge_{\substack{i=1\\i \neq j\\1\leqslant j-i+k \leqslant N}}^{N} \neg q_{i,j-i+k})]\} $$

% 6.
\item
\begin{enumerate}
	\item $P \land (Q \land R) \Leftrightarrow (P \land Q) \land R$ (see Table 1: 6-(a))


	\item $P \land (Q \lor R) \Leftrightarrow (P \land Q) \lor (P \land R)$ (see Table 2: 6-(b))


	\item $\neg(P \land Q) \Leftrightarrow \neg P \lor \neg Q$ (see Table 3: 6-(c))


	\item $\neg(P \land Q) \Leftrightarrow \neg P \lor \neg Q$ (see Table 4: 6-(d))

\end{enumerate}



% 7.
\item
\begin{enumerate}
	\item \textit{Smoke} $\Rightarrow$ \textit{Smoke} (see Table 5: 7-(a))

	Valid.


	% b
	\item \textit{Smoke} $\Rightarrow$ \textit{Fire} (see Table 6: 7-(b))
	
	Unsatisfiable.


	% c
	\item (\textit{Smoke} $\Rightarrow$ \textit{Fire}) $\Rightarrow$ ($\neg$\textit{Smoke} $\Rightarrow$ $\neg$\textit{Fire}) (see Table 7: 7-(c))

	Unsatisfiable.

	% d
	\item \textit{Smoke} $\lor$ \textit{Fire} $\lor$ $\neg$\textit{Fire} (see Table 8: 7-(d))
		
	Valid.


	% e
	\item ((\textit{Smoke} $\land$ \textit{Heat}) $\Rightarrow$ \textit{Fire}) $\Leftrightarrow$ ((\textit{Smoke} $\Rightarrow$ \textit{Fire}) $\lor$ (\textit{Heat} $\Rightarrow$ \textit{Fire}))

	Left:\\
	\begin{align*}
	& (S \land H) \Rightarrow F \\
	= & \neg (S \land H) \lor F \\
	= & \neg S \lor \neg H \lor F
	\end{align*}

	Right:\\
	\begin{align*}
	& (S \Rightarrow F) \lor (H \Rightarrow F) \\
	= & (\neg S \lor F) \lor (\neg H \lor F) \\
	= & \neg S \lor \neg H \lor F
	\end{align*}

	Left and Right are equivalent. So it's valid.



	% f
	\item \textit{Big} $\lor$ \textit{Dumb} $\lor$ (\textit{Dumb} $\Rightarrow$ \textit{Big})
	\begin{align*}
	& B \lor D \lor (D \Rightarrow B) \\
	= & B \lor D \lor (\neg D \lor B) \\
	= & B \lor (D \lor \neg D) 
	\end{align*}

	$D \lor \neg D$ is always \textit{True}, so that $B \lor (D \lor \neg D)$ is always \textit{True} too. So it's valid. 

\end{enumerate}



% 8. 
\item DPLL

$P \Rightarrow Q = \neg P \lor Q$ \\
$L \land M \Rightarrow P = \neg L \lor \neg M \lor P$ \\
$B \land L \Rightarrow M = \neg B \lor \neg L \lor M$ \\
$A \land P \Rightarrow L = \neg A \lor \neg P \lor L$ \\
$A \land B \Rightarrow L = \neg A \lor \neg B \lor L$ \\
$A$ \\
$B$

Step 1. \\
Clauses: $(\neg P \lor Q) \land (\neg L \lor \neg M \lor P) \land (\neg B \lor \neg L \lor M) \land (\neg A \lor \neg P \lor L) \land (\neg A \lor \neg B \lor L) \land A \land B$ \\
Find pure symbols: $Q$ \\
Model: $Q=$ \textit{True}

Step 2. \\
Clauses: $(\neg L \lor \neg M \lor P) \land (\neg B \lor \neg L \lor M) \land (\neg A \lor \neg P \lor L) \land (\neg A \lor \neg B \lor L) \land A \land B$ \\
Find unit symbols: $A,B$ \\
Model: $Q=$ \textit{True}, $A=$ \textit{True}, $B=$ \textit{True}

Step 3. \\
Clauses: $(\neg L \lor \neg M \lor P) \land (\neg L \lor M) \land (\neg P \lor L) \land L$ \\
Find pure symbols: None \\
Find unit symbols: $L$ \\
Model: $Q=$ \textit{True}, $A=$ \textit{True}, $B=$ \textit{True}, $L=$ \textit{True}

Step 4. \\
Clauses: $(\neg M \lor P) \land M$ \\
Find pure symbols: $P$ \\
Model: $Q=$ \textit{True}, $A=$ \textit{True}, $B=$ \textit{True}, $L=$ \textit{True}, $P=$ \textit{True}

Step 5. \\
Clauses: $M$ \\
Find unit symbols: $M$ \\
Model: $Q=$ \textit{True}, $A=$ \textit{True}, $B=$ \textit{True}, $L=$ \textit{True}, $P=$ \textit{True}, $M=$ \textit{True}

Step 6. \\
Done
\\
\\
Comparison between DPLL and FC algorithm:

They generate different traces based on $KB$. FC is data driven and DPLL has early termination. The complexity of DPLL would be much less than FC.


\end{enumerate}


\newpage

	\begin{table}[!h]
	\centering
	\caption{6-(a)}
	\begin{tabular}{|c|c|c|c|c|c|c|}
	\hline
	$P$ & $Q$ & $R$ & $Q \land R$ & $P \land Q$ & $P \land (Q \land R)$ & $(P \land Q) \land R$\\ \hline
	T & T & T & T & T & T & T  \\ \hline
	T & T & F & F & T & F & F  \\ \hline
	T & F & T & F & F & F & F  \\ \hline
	T & F & F & F & F & F & F  \\ \hline
	F & T & T & T & F & F & F  \\ \hline
	F & T & F & F & F & F & F  \\ \hline
	F & F & T & F & F & F & F  \\ \hline
	F & F & F & F & F & F & F  \\ \hline
	\end{tabular}
	\end{table}

	\begin{table}[!h]
	\centering
	\caption{6-(b)}
	\begin{tabular}{|c|c|c|c|c|c|c|c|}
	\hline
	$P$ & $Q$ & $R$ & $Q \lor R$ & $P \land Q$ & $P \land R$ & $P \land (Q \lor R)$ & $(P \land Q) \lor (P \land R)$\\ \hline
	T & T & T & T & T & T & T & T  \\ \hline
	T & T & F & T & T & F & T & T  \\ \hline
	T & F & T & T & F & T & T & T  \\ \hline
	T & F & F & F & F & F & F & F  \\ \hline
	F & T & T & T & F & F & F & F  \\ \hline
	F & T & F & T & F & F & F & F  \\ \hline
	F & F & T & T & F & F & F & F  \\ \hline
	F & F & F & F & F & F & F & F  \\ \hline
	\end{tabular}
	\end{table}

	\begin{table}[!h]
	\centering
	\caption{6-(c)}
	\begin{tabular}{|c|c|c|c|c|c|c|}
	\hline
	$P$ & $Q$ & $\neg P$ & $\neg Q$ & $P \land Q$ & $\neg(P \land Q)$ & $\neg P \lor \neg Q$\\ \hline
	T & T & F & F & T & F & F \\ \hline
	T & F & F & T & F & T & T \\ \hline
	F & T & T & F & F & T & T \\ \hline
	F & F & T & T & F & T & T \\ \hline
	\end{tabular}
	\end{table}	

	\begin{table}[!h]
	\centering
	\caption{6-(d)}
	\begin{tabular}{|c|c|c|c|c|c|c|c|}
	\hline
	$P$ & $Q$ & $\neg P$ & $\neg Q$ & $P \land Q$ & $\neg P \lor \neg Q$ & $P \Leftrightarrow Q$ & $(P \land Q) \lor (\neg P \lor \neg Q)$ \\ \hline
	T & T & F & F & T & F & T & T \\ \hline
	T & F & F & T & F & F & F & F \\ \hline
	F & T & T & F & F & F & F & F \\ \hline
	F & F & T & T & F & T & T & T \\ \hline
	\end{tabular}
	\end{table}		



	\begin{table}[h]
	\centering
	\caption{7-(a)}
	\begin{tabular}{|c|c|c|}
	\hline
	$S$ & $\neg S$ & $S\Rightarrow S (\neg S \lor S)$  \\ \hline
	T & F & T \\ \hline
	F & T & T \\ \hline
	\end{tabular}
	\end{table}	
	

	\begin{table}[h]
	\centering
	\caption{7-(b)}
	\begin{tabular}{|c|c|c|c|}
	\hline
	$S$ & $F$ & $\neg S$ & $S\Rightarrow F (\neg S \lor F)$  \\ \hline
	T & T & F & T\\ \hline
	T & F & F & F\\ \hline
	F & T & T & T\\ \hline
	F & F & T & T\\ \hline
	\end{tabular}
	\end{table}	
	

	\begin{table}[h]
	\centering
	\caption{7-(c)}
	\begin{tabular}{|c|c|c|c|c|c|c|}
	\hline
	$S$ & $F$ & $\neg S$ & $\neg F$ & $S\Rightarrow F$ & $\neg S \Rightarrow \neg F$ & $(S\Rightarrow F)\Rightarrow (\neg S \Rightarrow \neg F)$ \\ \hline
	T & T & F & F & T & T & T\\ \hline
	T & F & F & T & F & T & T\\ \hline
	F & T & T & F & T & F & F\\ \hline
	F & F & T & T & T & T & T\\ \hline
	\end{tabular}
	\end{table}	
	
	
	\begin{table}[h]
	\centering
	\caption{7-(d)}
	\begin{tabular}{|c|c|c|c|}
	\hline
	$S$ & $F$ & $\neg F$ & $S \lor F \lor \neg F$  \\ \hline
	T & T & F & T\\ \hline
	T & F & T & T\\ \hline
	F & T & F & T\\ \hline
	F & F & T & T\\ \hline
	\end{tabular}
	\end{table}	


\end{document}